\documentclass[letterpaper,10pt]{article}
\usepackage{fullpage}

\title{Dytan Manual}

%opening
\begin{document}
\maketitle

\tableofcontents

\begin{abstract}
Dytan is a dynamic taint analysis framework that allows users to
implement different kinds of dynamic taint analyses. This file
discusses the features of Dytan that can be accessed through the
high-level XML interface. The use of Dytan's core without leveraging
the interface is described in file internals.txt
\end{abstract}

\section{Prerequisites}

Dytan is based on PIN version 2.2-13635-gcc.4.0.0-ia32-linux. To
download and install the right version of PIN, users should execute
script \verb|setup-pin.sh|, which is provided with the distribution.
Because of recent changes in the Linux kernel, this version of PIN
does not work correctly on the most recent Linux distributions. We use
Linux kernel version 2.6.15. More precisely, we recommend users to
install Dytan on a machine running Ubuntu 6.06, which is the platform
on which it was tested.

To be able to compile Dytan, it is necessary to install the libxml2
development library (package libxml2-dev for Debian/Ubuntu users).

Control-flow based taint propagation leverages postdominance
information computed by one of Dytan's modules. In the presence of
indirect jumps in the code, postdominance information is not computed
and a special basic block is added in the position where the jump is.
To alleviate this issue, it is recommended to pre-process the source
code of the programs under analysis using CIL, whenever possible. CIL
eliminates some of the constructs that cause indirect jumps in the
code, such as\verb|switch| statements.

\section{Dytan's features and configuration}

The current implementation of Dytan's interface is incomplete. As of
now, Dytan allows users to:

\begin{itemize}
 \item Taint specific arguments of a function.
 \item Taint arbitrary memory ranges (from within a program).
 \item Taint data originating from a file
 \item Taint data originating from a network connection
\end{itemize}
Dytan's interface does not support yet function return values.

\subsection{Dytan's configuration}

Before running it, Dytan must be configured by modifying the
configuration file config.xml. The configuration file contains
information about sources, propagation policies, and sinks. They allow to define whether propagation
should occur only through data dependences or through both data and
control dependences. (Note that our current implementation accounts
only for direct control dependences, that is, dependences due to the
fact that something is not executed are not considered.) In addition 
to the propagation policies, the taint source can be seleted to be
a single file or multiple files by populating the \verb|<file>| tags. Dytan
can be configured to use any number of taint marks by filling the
\verb|<taint-marks>| tag in the configuration file. For tainting data
coming from the filesystem or the network, Dytan can be configured
to taint an entire 'read' at a time or taint the individual bytes
in the read separately by specifiying the \verb|<granularity>| tag in
the configuration file. The \verb|<granularity>| tag takes two possible
values \verb|PerRead| or \verb|PerByte|. Sample configuration files are provided
in the root folder of Dytan as well as in the example folders.

To taint data from a network connection, the configuration file needs
to have the tags in the following manner under the \verb|<sources>| tag.
\begin{verbatim}
<source type="network">
    <host>127.0.0.1</host>
    <port>80</port>
    <granularity>PerRead</granularity>
</source>
\end{verbatim}

Similarly, for tainting a file, the config file will look like:
\begin{verbatim}
<source type="path">
    <file>sample/wc/a.txt</file>
    <granularity>PerByte</granularity>
</source>
\end{verbatim}

There can be multiple \verb|<source>| entries in the configuration file.
It should be noted that the file path specified should be exactly
what the program internally uses.

\subsection{How to taint specific arguments of a function}

To taint function arguments, users must modify the functions in file
\verb|taint_func_args.cpp|. After modifying the file, Dytan must be
recompiled by invoking \verb|make| in the root directory of the
distribution.  As an example, the version of the file in the
distribution contains code to taint arguments to the \verb|main| function.
Please refer to that file for a better understanding of the next
paragraph, which describes how to modify file \verb|taint_func_args.cpp|.

The string array \verb|taint_function| should contain the names of the
routines whose parameters are to be tainted. In function
\verb|taint_routines(RTN, void *)|, there should be an if block for each of
the functions in the \verb|taint_function| array. Each of these blocks
should use PIN API to insert a call to a suitably defined wrapper
before the original function is executed. This is accomplished by
using PIN API function \verb|RTN_InsertCall|. This function takes the
following arguments:

\begin{verbatim}

- RTN rtn: the original function.
- IPOINT_BEFORE: tells PIN to insert the call before rtn's entry.
- AFUNPTR(<wrapper name>): specifies the wrapper function's name.
- Assuming that rtn has n arguments:
   - IARG_FUNCARG_ENTRYPOINT_REFERENCE, 0,
   - IARG_FUNCARG_ENTRYPOINT_REFERENCE, 1,
   - ...
   - IARG_FUNCARG_ENTRYPOINT_REFERENCE, n,
- IARG_END: tells RTN_InsertCall that there are no more parameters.
 
\end{verbatim}

The wrapper function is the place where the tainting actually occurs.
It takes n arguments of type \verb|ADDRINT *|, where n is the number of
arguments of the original function. For each parameter to be tainted,
the wrapper calls Dytan's memory tainting function, which takes three
parameters: the starting address of the parameter's memory location,
the size of the parameter in bytes, and the numeric value of the taint
mark to be associated with the parameter.  See function
\verb|main_wrapper_func| in file \verb|taint_func_args.cpp| for an example of a
wrapper.

\subsection{How to taint data originating from a file or from a network connection}

Dytan can taint data being read from a file or from a network socket.
The file or the network connection needs to be specified in the
configuration file in the \verb|sources| section as explained in the Configuration
section. The example folders in the Dytan distribution contain sample 
configuration files and code. At the end of execution of a program,
the taint markings will be dumped in "taint-log.log".


\subsection{How to taint arbitrary memory ranges (from within a program)}

To taint arbitrary memory ranges from within your program, make a call
to function
\begin{verbatim}
  DYTAN_tag(ADDRINT start_address, size_t size, char * name)
\end{verbatim}
where size is the size of the memory to be tainted and name is the
string to be associated with this taint mark.

Analogously, to display taint marks at a particular memory location
from within your program, make a call to function
\begin{verbatim}
  DYTAN_display(ADDRINT start_address, size_t size, char *fmt)
\end{verbatim}
where fmt is the format in which the taint marks should be displayed.

The sample program provided with this distribution, available in
directory \verb|sample/wc|, makes the above calls for tainting memory and
displaying memory taint marks.  See file internals.txt for
instructions on how to compile the sample program.


\section{Running Dytan}

After configuring and recompiling Dytan, it can be run by executing:
\begin{verbatim}
./run-dytan <program_name>
\end{verbatim}

For example, to run Dytan on the provided example, go to the root
directory of the distribution and run
\begin{verbatim}
./run-dytan sample/wc/wc sample/wc/a.txt
\end{verbatim}

\section{Internals}

See internals.txt for details on the internals of Dytan.

\end{document}
